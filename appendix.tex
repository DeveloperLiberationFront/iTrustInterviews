\documentclass[10pt,journal,compsoc]{IEEEtran}

\begin{document}

\section*{Recruitment Email}
Hi [Name],

My name is Justin Smith. I am a PhD student working with Dr. Murphy-Hill on doing a study with iTrust.
I am looking for students who have worked on this system in the past. Participation in this study will help
improve the design of security related software engineering tools.

The study will take approximately 45 -- 60 minutes to complete. Participants must perform the study at
NC State University, Room [TBD]. Familiarity with iTrust is required.

If you have any questions about the study or would like to schedule a time to participate, please respond
to me at this email address (jssmit11@ncsu.edu).
Participants must be 18 years of age or older.

\noindent Thanks!

\noindent Justin Smith 

% you can choose not to have a title for an appendix
% if you want by leaving the argument blank
\section*{Procedure}

\textbf{Introductory Script:} This experiment is designed to evaluate how you approach security vulnerabilities. This session
will be recorded using screen and audio capture software. I will present you with some code that
FindBugs has analyzed and ask you to explain what is going on.

This experiment is not designed to test your ability as a developer. Instead, we are attempting to
understand how you approach code vulnerabilities. During the experiment, you can use any web
resource. I have opened a web browser, which you are welcome to use at any point.

For the purposes of this experiment, imagine that you are in charge of security for iTrust, which
is nearing release. You must justify any proposed code changes that you make. Currently, the
software may contain security vulnerabilities. FindBugs has been configured to direct you to 4
areas of concern. For each area, I'll ask you about 10 questions.

Do you have any questions before we begin?

1) Navigate to first warning message

2) Read through the code/warning message

3) Try to think aloud. So, say any questions or thoughts that cross your mind regardless of
how relevant you think they are. 

\textbf{Experimental Questions:} 
\begin{itemize}
\item Can you explain what this warning is trying to tell you?

\item What are you exploring/trying to figure out right now?

\item What information do you need to proceed?

\item For the sake of time, explain to me what you would keep doing?

\item Would you normally do this in your work?

\item How does the wording effect the process?

\item Based on your understanding of this error message, what is the percentage likelihood you would modify this section of the code?

\item How would you fix the problem? Where would you start?

\item On a scale of 1-5, how confident are you in your understanding of this error message?

\item On a scale of 1-5, how confident are you that you are making the correct judgment?

\item What part of the message is most helpful?

\item What information would you like to see added to the error message?

\item Look at the error message one last time. Is there any information that you initially disregarded?

\item Anything else I should know?
\end{itemize}


\textbf{Post-Experiment Questions:}
\begin{itemize}
\item What is your current job title (if student, indicate so here)
\item How many years have you been programming?
\item How many years of professional programming experience do you have?
\item Over the last year, about how many hours per week would you say you spend programming, on
average?
\item When programming, do you typically use security tools? Y/N
\begin{itemize}
	\item If Y, Which tool(s) do you use? Why do you use these security tool?
	\item If N, can you tell me the reasons why you don't use security tools?
\end{itemize}
\item On a scale form 1 to 5, how familiar are you with security vulnerabilities?
(1 = not at all, to 5 = very familiar)
\item In general how effectively did the FindBugs communicate information to you? What did you or didn't you like?
\item What security related questions could the tool answer better?
\end{itemize}

\section*{Categorized Questions}
Here we present the full list of questions we identified and categorized. 
The \textit{\#Y.X} notation corresponds with the tags used in strategy and assumption analysis.


\noindent\textbf{Developer Planning and Self-Reflection \#1.X} \\
	What was I looking for?  \#1.1 \\
	What do I know now? \#1.2 \\
	What should I do first? \#1.3 \\
	Do I understand? \#1.4 \\
	Have I seen this before?  \#1.5 \\
	What was that again? \#1.6 \\
	What's the next thing I should be doing?  \#1.7 \\
	Where am I (in the code)?  \#1.8 \\
	Is this worth my time? \#1.9 \\
	Is this my responsibility?  \#1.10 \\
	Why do I care? \#1.11 \\
	Have I used this package before? \#1.12 \\
	Am I making the right decision for my code? \#1.13 \\
	How long have we been talking? \#1.14 \\
\textbf{Vulnerability Severity and Rank \#2.X} \\
	How serious is this vulnerability? \#2.1 \\
	Are all these vulnerabilities the same severity? \#2.2 \\
	How do the rankings compare? \#2.3 \\
	What do the vulnerability rankings mean? \#2.4 \\
\textbf{Data Storage and Flow \#3.X} \\
	How is data put into this variable? \#3.1 \\
	Does data from this method/code travel to the database? \#3.2 \\
	Where does this information/data go? \#3.3 \\
	How do I find where the information travels? \#3.4 \\
	How does the information change as it travels through the program? \#3.5 \\
	What does this variable contain? \#3.6 \\
	Is any of the data malicious?  \#3.7 \\
	Where is the data coming from? \#3.8 \\
	What are the contents of this SQL file? \#3.9 \\
	Where along the data pipeline is this method? \#3.10 \\
	Where along the pipeline do you want to make sure the data is secure? \#3.11 \\
\textbf{Application Context/Usage \#4.X} \\
	What is this method/variable used for in the program? \#4.1 \\
	Are we handling secure data in this context? \#4.2 \\
	What is the context of this vulnerability/code? \#4.3 \\
	How does the system work? \#4.4 \\
	Will usage of this method change? \#4.5 \\
	Is this method/variable ever being used? \#4.6 \\
	Is this code used to test the program/functionality? \#4.7 \\
	Does test utils get sent to production code? \#4.8 \\
	How many layers of code/security have we passed through before reaching this point? \#4.9 \\
\textbf{Control Flow and Call Information \#5.X} \\
	What is the call hierarchy?  \#5.1 \\
	How can I get calling information? \#5.2 \\
	Who can call this? \#5.3 \\
	Where is the method being called? \#5.4 \\
	What causes this to be called? \#5.5 \\
	Are all calls coming from the same class? \#5.6 \\
	What gets called when this method gets called? \#5.7 \\
	How often is this code called?  \#5.8 \\
	Is this called from another less secure API?  \#5.9 \\
	How frequently does this method get called?  \#5.10 \\
	How many calls does it take to reach the vulnerability?  \#5.11 \\
	What are the parameters called?  \#5.12 \\
	Is the entire set of possible parameters known in advance?  \#5.13 \\
\textbf{Resources and Documentation \#6.X} \\
	What type of information does this resource link me to? \#6.1 \\
	What is the documentation?  \#6.2 \\
	Can my team members/resources provide me with more information? \#6.3 \\
	Where can I get more information? \#6.4 \\
	What information is in the documentation? \#6.5 \\
	Is this a reliable/trusted resource? \#6.6 \\
	How do resources prevent or resolve this? \#6.7 \\
	How should I word my search to get the right information? \#6.8 \\
	Is this the OWASP site? \#6.9 \\
	Does the method have a javadoc? \#6.10 \\
\textbf{Understanding Alternative Fixes and Approaches \#7.X} \\
	Why should I use this alternative method/approach to fix the vulnerability? \#7.1 \\
	What are the alternatives for fixing this? \#7.2 \\
	Does the alternative function the same as what I'm currently using? \#7.3 \\
	When should I use the alternative?  \#7.4 \\
	Is the alternative slower?  \#7.5 \\
	Are there other considerations to make when using the alternative(s)? \#7.6 \\
	How does my code compare to the alternative code in the example I found? \#7.7 \\
	Is secure random good?  \#7.8 \\
	Is prepared statement a Java thing or a 3rd party library?  \#7.9 \\
	How secure is secure random?  \#7.10 \\
\textbf{Code Background and Functionality \#8.X} \\
	Who wrote this code? \#8.1 \\
	Why is this code needed? \#8.2  \\
	Is this library code? \#8.3 \\
	Are there tests for this code?  \#8.4 \\
	Why was this code written this way? \#8.5 \\
	What does this code do? \#8.6 \\
	Is this code doing anything? \#8.7 \\
	How much effort was put into this code? \#8.8 \\
	Why are we using this API in the code? \#8.9 \\
	Is there a list of files we are going to iterate over?  \#8.10 \\
	Is rand static or final?  \#8.11 \\
	Why are we collecting this information in the first place?  \#8.12 \\
	Is there a variable called ``dir'' at the top?  \#8.13 \\
	How does the code accomplish design goals?  \#8.14 \\
	What is the name of the current class?  \#8.15 \\
	Could the pointer be null?  \#8.16 \\
	Does it throw an exception instead  \#8.17 \\
\textbf{Locating Information \#9.X} \\
	Where is this used in the code? \#9.1 \\
	Where are other similar pieces of code? \#9.2 \\
	Where is this artifact?  \#9.3 \\
	Is this artifact located in this class? \#9.4 \\
	Where is this method defined?  \#9.5 \\
	Where is this class?  \#9.6 \\
	Where is the next occurrence of this variable?  \#9.7 \\
	How do I track this information in the code? \#9.8 \\
	How do I navigate to other open files? \#9.9 \\
	Where is this class instantiated  \#9.10 \\
	Where was the class instantiated  \#9.11 \\
\textbf{Assessing the Application of the Fix \#10.X} \\
	How hard is it to apply a fix to this code? \#10.1 \\
	How do I use this fix in my code? \#10.2 \\
	How do I fix this vulnerability? \#10.3 \\
	Is there a quick fix for automatically applying a fix? \#10.4 \\
	Will the code work the same after I apply the fix? \#10.5 \\
	Can these fix suggestions be applied to my code? \#10.6 \\
	Will the error go away when I apply this fix? \#10.7 \\
	Does the code stand up to additional tests prior to/after applying the fix?  \#10.8 \\
	What other changes do I need to make to apply this fix? \#10.9 \\
\textbf{Preventing and Understanding Potential Attacks \#11.X} \\
	How can this vulnerability lead to an attack?  \#11.1 \\
	How can I replicate an attack that exploits this vulnerability? \#11.2 \\
	Why is this a vulnerability? \#11.3 \\
	What are the possible attacks that could occur? \#11.4  \\
	How can I prevent this attack? \#11.5 \\
	How should I address this problem?  \#11.6 \\
	What is the problem (potential attack)? \#11.7 \\
	Is this a real vulnerability? \#11.8 \\
	How do I find out if this is a real vulnerability? \#11.9 \\
	How much value would you get from randomly injecting stuff? \#11.10 \\
	Is releaseForms handling data securely? \#11.11 \\
\textbf{Relationship Between Vulnerabilities \#12.X} \\
	Does this other piece code have the same vulnerability as the code I'm working with?  \#12.1 \\
	Are all the vulnerabilities related in my code? \#12.2 \\
	Are all of these notifications vulnerabilities?  \#12.3 \\
\textbf{End-User Interaction \#13.X} \\
	Is there input coming from the user? \#13.1 \\
	Does the user have access to this code? \#13.2 \\
	Does user input get validated/sanitized? \#13.3 \\
\textbf{Notification Text \#14.X} \\
	What is the relatinoship between the error message and the code? \#14.1 \\
	What code caused this error message to occur? \#14.2 \\
	What does the error message say? \#14.3 \\
\textbf{Understanding Concepts \#15.X} \\
	What is the term for this concept? \#15.1 \\
	Do these words have special meaning related to this concept/problem? \#15.2 \\
	How does this concept work? \#15.3 \\
	What is this concept? \#15.4 \\
	Is Java's true/false considered enum? \#15.5 \\
	What is the nomenclature for testing suites? \#15.6 \\
\textbf{Confirming Expectations  \#16.X} \\
	Is this doing what I expect it to? \#16.1 \\
\textbf{Uncategorized  \#17.X} \\
	Have I exhausted all references in this method? \#17.1 \\
	Is this method deprecated? \#17.2 \\
	What does the constructor and datatype look like? \#17.3 \\
	Was the author of this code aware of security issues? \#17.4 \\
	Has the package I'm using been tested? \#17.5 \\
	Is it done the same way in Python? \#17.6 \\
	Are there non-malicious ways the code could break? \#17.7 \\
	How can I transfer text from the IDE to the browser? \#17.8 \\
	What parts of the code do you trust with this data? \#17.9 \\
	What do you want to trust? \#17.10 \\
\textbf{Understanding and Interacting with Tools \#18.X} \\
	Why is the tool complaining? \#18.1 \\
	What is the tool output telling me?  \#18.2 \\
	Can I verify the information the tool provides? \#18.3 \\
	What is the tool keybinding?  \#18.4 \\
	What is the tool's confidence? \#18.5 \\
	What tool do I need for this?  \#18.6 \\
	How is the information presented by the tool organized?  \#18.7 \\
	What is the related stack trace for this method? \#18.8 \\
	How can I annotate that these strings have been escaped and the tool should ignore the warning? \#18.9 \\
\textbf{Discarded (5) \#Discard}

\end{document}
