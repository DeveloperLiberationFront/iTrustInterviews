\documentclass[10pt,journal,compsoc]{IEEEtran}

\begin{document}

\section*{Recruitment Email}
Hi [Name],

My name is Justin Smith. I am a PhD student working with Dr. Murphy-Hill on doing a study with iTrust.
I am looking for students who have worked on this system in the past. Participation in this study will help
improve the design of security related software engineering tools.

The study will take approximately 45 -- 60 minutes to complete. Participants must perform the study at
NC State University, Room [TBD]. Familiarity with iTrust is required.

If you have any questions about the study or would like to schedule a time to participate, please respond
to me at this email address (jssmit11@ncsu.edu).
Participants must be 18 years of age or older.

\noindent Thanks!

\noindent Justin Smith 

% you can choose not to have a title for an appendix
% if you want by leaving the argument blank
\section*{Procedure}

\textbf{Introductory Script:} This experiment is designed to evaluate how you approach security vulnerabilities. This session
will be recorded using screen and audio capture software. I will present you with some code that
FindBugs has analyzed and ask you to explain what is going on.

This experiment is not designed to test your ability as a developer. Instead, we are attempting to
understand how you approach code vulnerabilities. During the experiment, you can use any web
resource. I have opened a web browser, which you are welcome to use at any point.

For the purposes of this experiment, imagine that you are in charge of security for iTrust, which
is nearing release. You must justify any proposed code changes that you make. Currently, the
software may contain security vulnerabilities. FindBugs has been configured to direct you to 4
areas of concern. For each area, I'll ask you about 10 questions.

Do you have any questions before we begin?

1) Navigate to first warning message

2) Read through the code/warning message

3) Try to think aloud. So, say any questions or thoughts that cross your mind regardless of
how relevant you think they are. 

\textbf{Experimental Questions:} 
\begin{itemize}
\item Can you explain what this warning is trying to tell you?

\item What are you exploring/trying to figure out right now?

\item What information do you need to proceed?

\item For the sake of time, explain to me what you would keep doing?

\item Would you normally do this in your work?

\item How does the wording effect the process?

\item Based on your understanding of this error message, what is the percentage likelihood you would modify this section of the code?

\item How would you fix the problem? Where would you start?

\item On a scale of 1-5, how confident are you in your understanding of this error message?

\item On a scale of 1-5, how confident are you that you are making the correct judgment?

\item What part of the message is most helpful?

\item What information would you like to see added to the error message?

\item Look at the error message one last time. Is there any information that you initially disregarded?

\item Anything else I should know?
\end{itemize}


\textbf{Post-Experiment Questions:}
\begin{itemize}
\item What is your current job title (if student, indicate so here)
\item How many years have you been programming?
\item How many years of professional programming experience do you have?
\item Over the last year, about how many hours per week would you say you spend programming, on
average?
\item When programming, do you typically use security tools? Y/N
\begin{itemize}
	\item If Y, Which tool(s) do you use? Why do you use these security tool?
	\item If N, can you tell me the reasons why you don't use security tools?
\end{itemize}
\item On a scale form 1 to 5, how familiar are you with security vulnerabilities?
(1 = not at all, to 5 = very familiar)
\item In general how effectively did the FindBugs communicate information to you? What did you or didn't you like?
\item What security related questions could the tool answer better?
\end{itemize}

\end{document}
