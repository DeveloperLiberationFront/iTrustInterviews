%% iTrustInterviews.tex
%% V1.4
%% 2012/12/27
%% by Michael Shell
%% See:
%% http://www.michaelshell.org/
%% for current contact information.
%%
%% This is a skeleton file demonstrating the use of IEEEtran.cls
%% (requires IEEEtran.cls version 1.8 or later) with an IEEE conference paper.
%%
%% Support sites:
%% http://www.michaelshell.org/tex/ieeetran/
%% http://www.ctan.org/tex-archive/macros/latex/contrib/IEEEtran/
%% and
%% http://www.ieee.org/

%%*************************************************************************
%% Legal Notice:
%% This code is offered as-is without any warranty either expressed or
%% implied; without even the implied warranty of MERCHANTABILITY or
%% FITNESS FOR A PARTICULAR PURPOSE! 
%% User assumes all risk.
%% In no event shall IEEE or any contributor to this code be liable for
%% any damages or losses, including, but not limited to, incidental,
%% consequential, or any other damages, resulting from the use or misuse
%% of any information contained here.
%%
%% All comments are the opinions of their respective authors and are not
%% necessarily endorsed by the IEEE.
%%
%% This work is distributed under the LaTeX Project Public License (LPPL)
%% ( http://www.latex-project.org/ ) version 1.3, and may be freely used,
%% distributed and modified. A copy of the LPPL, version 1.3, is included
%% in the base LaTeX documentation of all distributions of LaTeX released
%% 2003/12/01 or later.
%% Retain all contribution notices and credits.
%% ** Modified files should be clearly indicated as such, including  **
%% ** renaming them and changing author support contact information. **
%%
%% File list of work: IEEEtran.cls, IEEEtran_HOWTO.pdf, bare_adv.tex,
%%                    bare_conf.tex, bare_jrnl.tex, bare_jrnl_compsoc.tex,
%%                    bare_jrnl_transmag.tex
%%*************************************************************************

% *** Authors should verify (and, if needed, correct) their LaTeX system  ***
% *** with the testflow diagnostic prior to trusting their LaTeX platform ***
% *** with production work. IEEE's font choices can trigger bugs that do  ***
% *** not appear when using other class files.                            ***
% The testflow support page is at:
% http://www.michaelshell.org/tex/testflow/



% Note that the a4paper option is mainly intended so that authors in
% countries using A4 can easily print to A4 and see how their papers will
% look in print - the typesetting of the document will not typically be
% affected with changes in paper size (but the bottom and side margins will).
% Use the testflow package mentioned above to verify correct handling of
% both paper sizes by the user's LaTeX system.
%
% Also note that the "draftcls" or "draftclsnofoot", not "draft", option
% should be used if it is desired that the figures are to be displayed in
% draft mode.
%
\documentclass[conference]{IEEEtran}
% Add the compsoc option for Computer Society conferences.
%
% If IEEEtran.cls has not been installed into the LaTeX system files,
% manually specify the path to it like:
% \documentclass[conference]{../sty/IEEEtran}





% Some very useful LaTeX packages include:
% (uncomment the ones you want to load)


% *** MISC UTILITY PACKAGES ***
%
%\usepackage{ifpdf}
% Heiko Oberdiek's ifpdf.sty is very useful if you need conditional
% compilation based on whether the output is pdf or dvi.
% usage:
% \ifpdf
%   % pdf code
% \else
%   % dvi code
% \fi
% The latest version of ifpdf.sty can be obtained from:
% http://www.ctan.org/tex-archive/macros/latex/contrib/oberdiek/
% Also, note that IEEEtran.cls V1.7 and later provides a builtin
% \ifCLASSINFOpdf conditional that works the same way.
% When switching from latex to pdflatex and vice-versa, the compiler may
% have to be run twice to clear warning/error messages.






% *** CITATION PACKAGES ***
%
%\usepackage{cite}
% cite.sty was written by Donald Arseneau
% V1.6 and later of IEEEtran pre-defines the format of the cite.sty package
% \cite{} output to follow that of IEEE. Loading the cite package will
% result in citation numbers being automatically sorted and properly
% "compressed/ranged". e.g., [1], [9], [2], [7], [5], [6] without using
% cite.sty will become [1], [2], [5]--[7], [9] using cite.sty. cite.sty's
% \cite will automatically add leading space, if needed. Use cite.sty's
% noadjust option (cite.sty V3.8 and later) if you want to turn this off
% such as if a citation ever needs to be enclosed in parenthesis.
% cite.sty is already installed on most LaTeX systems. Be sure and use
% version 4.0 (2003-05-27) and later if using hyperref.sty. cite.sty does
% not currently provide for hyperlinked citations.
% The latest version can be obtained at:
% http://www.ctan.org/tex-archive/macros/latex/contrib/cite/
% The documentation is contained in the cite.sty file itself.





% *** GRAPHICS RELATED PACKAGES ***
%
\ifCLASSINFOpdf
  % \usepackage[pdftex]{graphicx}
  % declare the path(s) where your graphic files are
  % \graphicspath{{../pdf/}{../jpeg/}}
  % and their extensions so you won't have to specify these with
  % every instance of \includegraphics
  % \DeclareGraphicsExtensions{.pdf,.jpeg,.png}
\else
  % or other class option (dvipsone, dvipdf, if not using dvips). graphicx
  % will default to the driver specified in the system graphics.cfg if no
  % driver is specified.
  % \usepackage[dvips]{graphicx}
  % declare the path(s) where your graphic files are
  % \graphicspath{{../eps/}}
  % and their extensions so you won't have to specify these with
  % every instance of \includegraphics
  % \DeclareGraphicsExtensions{.eps}
\fi
% graphicx was written by David Carlisle and Sebastian Rahtz. It is
% required if you want graphics, photos, etc. graphicx.sty is already
% installed on most LaTeX systems. The latest version and documentation
% can be obtained at: 
% http://www.ctan.org/tex-archive/macros/latex/required/graphics/
% Another good source of documentation is "Using Imported Graphics in
% LaTeX2e" by Keith Reckdahl which can be found at:
% http://www.ctan.org/tex-archive/info/epslatex/
%
% latex, and pdflatex in dvi mode, support graphics in encapsulated
% postscript (.eps) format. pdflatex in pdf mode supports graphics
% in .pdf, .jpeg, .png and .mps (metapost) formats. Users should ensure
% that all non-photo figures use a vector format (.eps, .pdf, .mps) and
% not a bitmapped formats (.jpeg, .png). IEEE frowns on bitmapped formats
% which can result in "jaggedy"/blurry rendering of lines and letters as
% well as large increases in file sizes.
%
% You can find documentation about the pdfTeX application at:
% http://www.tug.org/applications/pdftex


% correct bad hyphenation here
\hyphenation{op-tical net-works semi-conduc-tor}

\usepackage{url}

\begin{document}
%
% paper title
% can use linebreaks \\ within to get better formatting as desired
% Do not put math or special symbols in the title.
\title{iTrust Interviews}


% author names and affiliations
% use a multiple column layout for up to three different
% affiliations
\author{\IEEEauthorblockN{Justin Smith and Brittany Johnson}
\IEEEauthorblockA{North Carolina State University\\
Raleigh, North Carolina}}


% conference papers do not typically use \thanks and this command
% is locked out in conference mode. If really needed, such as for
% the acknowledgment of grants, issue a \IEEEoverridecommandlockouts
% after \documentclass

% for over three affiliations, or if they all won't fit within the width
% of the page, use this alternative format:
% 
%\author{\IEEEauthorblockN{Michael Shell\IEEEauthorrefmark{1},
%Homer Simpson\IEEEauthorrefmark{2},
%James Kirk\IEEEauthorrefmark{3}, 
%Montgomery Scott\IEEEauthorrefmark{3} and
%Eldon Tyrell\IEEEauthorrefmark{4}}
%\IEEEauthorblockA{\IEEEauthorrefmark{1}School of Electrical and Computer Engineering\\
%Georgia Institute of Technology,
%Atlanta, Georgia 30332--0250\\ Email: see http://www.michaelshell.org/contact.html}
%\IEEEauthorblockA{\IEEEauthorrefmark{2}Twentieth Century Fox, Springfield, USA\\
%Email: homer@thesimpsons.com}
%\IEEEauthorblockA{\IEEEauthorrefmark{3}Starfleet Academy, San Francisco, California 96678-2391\\
%Telephone: (800) 555--1212, Fax: (888) 555--1212}
%\IEEEauthorblockA{\IEEEauthorrefmark{4}Tyrell Inc., 123 Replicant Street, Los Angeles, California 90210--4321}}




% use for special paper notices
%\IEEEspecialpapernotice{(Invited Paper)}

% make the title area
\maketitle

% As a general rule, do not put math, special symbols or citations
% in the abstract
\begin{abstract}
To assist developers in creating and understanding secure code, we should understand their knowledge requirements when working with code. 
To enhance our understanding, we conducted 10 semi-structured interviews with software developers, observing their interactions with security vulnerabilities. 
From these interactions, we extracted questions developers asked while assessing the security of iTrust, a Java medical records software system. 
Participants asked X questions which we grouped into Y categories. RESULTS...
Tool IMPLICATIONS...

What topic?

\end{abstract}

% no keywords

% For peer review papers, you can put extra information on the cover
% page as needed:
% \ifCLASSOPTIONpeerreview
% \begin{center} \bfseries EDICS Category: 3-BBND \end{center}
% \fi
%
% For peerreview papers, this IEEEtran command inserts a page break and
% creates the second title. It will be ignored for other modes.
\IEEEpeerreviewmaketitle



\section{Introduction}
% no \IEEEPARstart
Widespread security vulnerabilities enable potential attackers to exploit vital software systems. 
Accordingly, developers concerned with the security of their systems increasingly stress the importance of identifying and eliminating vulnerabilities as early as possible.
Distributing a defective piece of software may compromise the security of the entire system.

Developers are concerned with the security of their applications, but lack adequate support for reasoning about the security of their systems.

Toolsmiths are equipped with an exhaustive set of research findings and guidelines to aid in the design of usable tools [CITE].
However, their tools go unused, in part because they do not help developers answer relevant questions [CITE]. 
Our work aims to inform toolsmiths of the security-related questions that developers ask.

General Problem
Related Stuff

What is the paper about
	What did we do

Contribution:
	Inform toolsmiths of knowledge requirements.

Structure

Assess knowledge requirements that software developers have.



% An example of a floating figure using the graphicx package.
% Note that \label must occur AFTER (or within) \caption.
% For figures, \caption should occur after the \includegraphics.
% Note that IEEEtran v1.7 and later has special internal code that
% is designed to preserve the operation of \label within \caption
% even when the captionsoff option is in effect. However, because
% of issues like this, it may be the safest practice to put all your
% \label just after \caption rather than within \caption{}.
%
% Reminder: the "draftcls" or "draftclsnofoot", not "draft", class
% option should be used if it is desired that the figures are to be
% displayed while in draft mode.
%
%\begin{figure}[!t]
%\centering
%\includegraphics[width=2.5in]{myfigure}
% where an .eps filename suffix will be assumed under latex, 
% and a .pdf suffix will be assumed for pdflatex; or what has been declared
% via \DeclareGraphicsExtensions.
%\caption{Simulation Results.}
%\label{fig_sim}
%\end{figure}

% Note that IEEE typically puts floats only at the top, even when this
% results in a large percentage of a column being occupied by floats.


% An example of a double column floating figure using two subfigures.
% (The subfig.sty package must be loaded for this to work.)
% The subfigure \label commands are set within each subfloat command,
% and the \label for the overall figure must come after \caption.
% \hfil is used as a separator to get equal spacing.
% Watch out that the combined width of all the subfigures on a 
% line do not exceed the text width or a line break will occur.
%
%\begin{figure*}[!t]
%\centering
%\subfloat[Case I]{\includegraphics[width=2.5in]{box}%
%\label{fig_first_case}}
%\hfil
%\subfloat[Case II]{\includegraphics[width=2.5in]{box}%
%\label{fig_second_case}}
%\caption{Simulation results.}
%\label{fig_sim}
%\end{figure*}
%
% Note that often IEEE papers with subfigures do not employ subfigure
% captions (using the optional argument to \subfloat[]), but instead will
% reference/describe all of them (a), (b), etc., within the main caption.


% An example of a floating table. Note that, for IEEE style tables, the 
% \caption command should come BEFORE the table. Table text will default to
% \footnotesize as IEEE normally uses this smaller font for tables.
% The \label must come after \caption as always.
%
%\begin{table}[!t]
%% increase table row spacing, adjust to taste
%\renewcommand{\arraystretch}{1.3}
% if using array.sty, it might be a good idea to tweak the value of
% \extrarowheight as needed to properly center the text within the cells
%\caption{An Example of a Table}
%\label{table_example}
%\centering
%% Some packages, such as MDW tools, offer better commands for making tables
%% than the plain LaTeX2e tabular which is used here.
%\begin{tabular}{|c||c|}
%\hline
%One & Two\\
%\hline
%Three & Four\\
%\hline
%\end{tabular}
%\end{table}


\section{Related Work}
Two types of related work... Introduce them!

\subsection{Technical Approaches}
Developers working on code use strategies and tools to help secure their systems. 
Many studies have assessed the effectiveness...

A significant body of work focuses on making static analysis tools more effective from a technical perspective, rather than more usable. 
For example, researchers approach improving tools from a variety of angles, including reducing the number false positives, identifying new types of vulnerabilities, and optimizing taint checking~\cite{jovanovic2006pixy, livshits2005finding}.


\subsection{Questions Papers}
Several studies have explored the knowledge requirements of developers, however few such studies have focused specifically on the knowledge requirements of developers when assessing security vulnerabilities~\cite{begel2014analyze, latoza2010hard, latoza2010developers}.


Questions asked Self reported vs Working with code


\section{Methodology}
We conducted semi-structured interviews with software developers.
Each interview included a five-minute briefing section, followed by encounters with four vulnerabilities.
All participants consented to have their session recorded using screen and audio capture software.
Finally, each interview concluded with several demographic and open-ended questions.

Subsection I details the interview process and subsection II describes question extraction and analysis.


\subsection{Research Questions}
\begin{itemize}
\item RQ1: What types of security related questions do developers ask?
\item RQ2: What types of security related questions do developers fail to answer?
\end{itemize}

\subsection{Study Design}
-Tool choice (list comparison)
\\
-Background on iTrust
\\
-In person
\\
-Vulnerabilities
\\
-Participant recruiting
            	Snowball sampling
            	Email recruitments
            	Familiarity with code base.
            	Stop when responses stabalized, not much new data
		Extraction time intensive
\\
-Participants
            	Experience
            	Students and Professionals
\\
-Interview Procedure
            	~1 hour. Appropriate length in pilots. Eclipse
\\-Interview Questions
\\

\subsubsection{Materials}
Participants used Eclipse to explore vulnerabilities in iTrust,\footnote{\url{agile.csc.ncsu.edu/iTrust/wiki/doku.php}} a Java medical records application that ensures the privacy and security of patient records according to the HIPAA statue.\footnote{\url{hhs.gov/ocr/privacy/}} 
Participants were equipped with an extended version FindBugs, Find Security Bugs.\footnote{\url{http://h3xstream.github.io/find-sec-bugs/}} 
We chose Find Security Bugs as a representative tool by comparing the available source code analysis tools listed by NIST, OWASP, and WASC.


\subsubsection{Participants}
We conducted semi-structured interviews with 10 software developers. 
We recruited participants from personal contacts and class rosters, using snowball sampling to find more participants that had experience with iTrust.

\subsubsection{Tasks}
To inform the design of our study, we conducted a series of seven preliminary pilot interviews, in which participants spent approximately 10-15 minutes with each vulnerability and showed signs of fatigue after 60 minutes.  
Accordingly, we presented each participant with four vulnerabilities. 
To cover a broader set of question topics, vulnerabilities were selected from four different FindBugs categories. 
Find Security Bugs identified three types of naturally occurring vulnerabilities, cross site scripting, path traversal, and predictable random vulnerabilities. 
We inserted a SQL injection vulnerability by making minimal modifications to one of the database access objects. 
Our modifications preserved the functionality of the original code and were based on examples of SQL injection presented on OWASP.



\subsection{Data Analysis}
OVERVIEW
What is grounded theory [CITE]. 

[CITE LETOVSKY]

First, we transcribed all the audio files using oTranscribe.\footnote{\url{otranscribe.com}}
Using a grounded theory approach, each of the interviews was analyzed by two of the authors for implicit and explicit questions. 
We also extracted demographic information, self-efficacy scores and whether or not the participant would modify the code. ?? The two question sets for each interview were then iteratively compared against each other until the authors reached agreement on the question sets. 
In the remainder of this section, we will detail the question extraction process, including the criteria used to determine which statements qualified as questions.
\subsubsection{Question Criteria}
Participants ask both explicit and implicit questions. We developed 5 criteria to assist in the uniform classification of participant statements. A statement was coded as a question only if it met one of the following criteria.

\begin{itemize}
\item \textbf{The participant explicitly asks a question.}
\\ \textit{EXAMPLE}
\item \textbf{The participant makes a statement and explores the validity of that statement.}
\\ \textit{EXAMPLE}
\item \textbf{The participant uses key words such as, "I assume," "I guess," or "I don't know."}
\\ \textit{EXAMPLE}
\item \textbf{The participant clearly expresses uncertainty over a statement.}
\\ \textit{EXAMPLE}
\item \textbf{The participant clearly expresses a knowledge requirement by describing plans to acquire information.}
\\ \textit{EXAMPLE}

\end{itemize}

\subsubsection{Question Extraction}
Using the criteria outlined in the previous section, each interview was independently coded for questions by two of the authors. 
When we identified a statement that satisfied one or more of the criteria, we marked the transcript, highlighting the participants original statement, and clarified the question being asked.

FIGURE OF WORD

To handle instances where the two coders disagreed, each interview was reviewed a second time by the two authors that initially coded it.
During the second review, the two reviewers examined each question statement, discussing the justification for each question based on the previously stated criteria.
If one of the reviewers did not agree that a question met at least one of the criteria, the question was removed from the question set. 
When reviewers did agree, we determined the  phrasing for the the question that was most strongly grounded in the interview artifacts. 
AGREEMENT NUMBERS Ranged from xx to xx and averaged across participatns to xx 


-Transcription
\\
-Question extracton
\\
-Card Sorting
\\
-Rater Reliability

\section{Results}
We extracted x questions, removing repeats, we came up with x questions

\section{Discussion}
Data suggests that tools should focus on XYZ

\section{Conclusion}
The conclusion goes here.




% conference papers do not normally have an appendix


% use section* for acknowledgement
\section*{Acknowledgment}


The authors would like to thank...


\bibliographystyle{IEEEtran}
% argument is your BibTeX string definitions and bibliography database(s)
\bibliography{iTrustInterviews}
%
% <OR> manually copy in the resultant .bbl file
% set second argument of \begin to the number of references
% (used to reserve space for the reference number labels box)




% that's all folks
\end{document}


